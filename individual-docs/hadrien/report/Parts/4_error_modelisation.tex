\documentclass[10pt,a4paper,twoside,twocolumn]{article}
%% Lots of packages !
\usepackage{etex}

%% Francisation
\usepackage[english]{babel}
\usepackage[T1]{fontenc}
\usepackage[utf8]{inputenc}
%\usepackage{textcomp}

%% Réglages généraux
\usepackage[left=1.5cm,right=1.5cm,top=2cm,bottom=2cm]{geometry}
\usepackage{fancyhdr}
\usepackage{setspace}
\usepackage{lscape}
%\usepackage{multicol}
\usepackage{makeidx}
\usepackage[clearempty]{titlesec}
\usepackage{cite}

%% Packages pour le texte
\usepackage{pifont}
\usepackage{eurosym}
\usepackage{soul}
\usepackage[normalem]{ulem}
\usepackage{fancybox}
\usepackage{boxedminipage}
\usepackage{enumerate}
\usepackage{verbatim}
\usepackage{moreverb}
\usepackage{listings}
\usepackage[table]{xcolor}

%% Packages pour les tableaux
\usepackage{array}
\usepackage{multirow}
\usepackage{tabularx}
\usepackage{longtable}

%% Packages pour les dessins
\usepackage{graphicx}
\usepackage{wrapfig}
%\usepackage{picins}
\usepackage{picinpar}
\usepackage{epic}
\usepackage{eepic}
\usepackage{tikz}
\usepackage{afterpage}
\usepackage{rotating}
\usepackage{float}
\usepackage{caption}

%% Packages pour les maths
\usepackage{amsmath}
\usepackage{amssymb}
\usepackage{dsfont}
\usepackage{mathrsfs}
\usepackage{bussproofs}
\usepackage[thmmarks,amsmath]{ntheorem}

%% Création de nouvelles commandes
%\usepackage{calc}
\usepackage{ifthen}
\usepackage{xspace}



\usepackage{url}
\usepackage{hyperref}
\usepackage{todonotes}
\usepackage{subcaption}
\usepackage[french,ruled,vlined,linesnumbered,algosection,dotocloa]{algorithm2e}
\usepackage{MnSymbol}

\usepackage{chngcntr}

\usepackage{standalone}
\usepackage{import}

\usepackage[affil-it]{authblk}


\usepackage{lipsum}












\numberwithin{equation}{subsection}


\newcommand*{\rootPath}{../}
\standalonetrue

\begin{document}
\section{Error modelisation}

Memory corruption related error can happen in different ways. Cosmic rays and
radiation have, for a long time, been suspected of creating random data
\todo{ref}. More rescently we realised that thoses error can also happen because
of hardware issues\todo{ref}.

Beyound the question of those corruptions causes, we are here focussing on the
impact such event have on our pipeline.

\subsection{Error classification}

Depending on a large number of factor, memory corruption can have very different
concequences. Such factor include many things from hardware components (ECC
memory one of the most well know mecanism against memory correption) to software
certification and computation redondancy.

As for the consequences, they can be divided into two different categories :
\begin{description}
	\item[Hard errors:] Memory corruption affect the system of the program flow
		will most likely cause dramatic errors such as the program stopping
		abruptly or the whole system failling. In such cases we do not get any
		results back, redering irrelevant the question of the result's validity.
		Some memory corruptions affecting critical data such as table indices also
		falls into this category.

	\item[Silent errors:] Memory corruption in some application's data may not
		cause any crash of the application while affecting the results if not
		corrected by the hardware. This is particularly the case of large array's
		contents like particles positions in our pipeline.
\end{description}

In this section we will try to caracterise silent errors' impact on our
pipeline's results.

\subsection{Error injection}

Simulating random memory corruption can be done by voluntarly modifying our
applications data by randomly flipping bits. While this isn't hard to do,
studying silent errors means we have to enshure that those random bit flips will
not cause hard errors.

Differents software quality implies different level of tolerence. For exemple
some code can handle particles positions being outside of our considered domain
while other might crash\todo{harderror ratio for tess ?}. Form here on we will
mostly focus on our AKDE implementation as it very resistant to such errors and
therefore more suitable to create and analyse silent errors.

In our model, simulating memory corruption is achieved by randomly modifying our
input data set. In real application this dataset would be provided by another
program and would have therefor been send through the network of saved on disk,
increasing the probability of memory corruption. This injection model will be
controlled by two parameters, on the one hand the number of bit flips and one
the second hand the weight of the potentially affected bits.

While the first parameter is used to simulate different degre of corruption, the
second one is used to study the impact of differents bit flip position. The
construction of IEEE floating point arithmetics\cite{Kahan1996} (IEEE 754) is
such different bit modification produce different arithmetic modification.

Studies have shown that in some HPC pipelines, some bits' position are critical,
the modification of those bits resulting in hard errors while some other bits' 
modifications are un noticeable as the resulting modification are bellow the
accuracy of floating computation\todo{ref - Leonardo's tech report}.

\subsubsection{Single error injection}

As a first step in our analysis of memory corruption's impact on density
estimator we will study the impact of single bitflip positions.

We are therefore going to modify, for various samplings of our density function,
the value of one randomly selected floating value by flipping it's $n$-th bit.
For single precision floating values this $n$ value varies between $0$ and $255$
as simple precision floating value are $32$ byte long. Once this mofication has
been done, the corrupted sampling is processed by the pipeline and compared to
uncorrupted results.

\subsubsection{Multiple error injection}

The second approch we are developping is to study the impact of memory
corruption rate on result's quality.

Our input data are sampling containing :
\begin{table}[h]
	\centering
	\begin{tabular}{rlcrl}
		$2\times10^5$	& particles	&=	& $6\times10^5$			& floats	\\
									&						&= 	& $1,92\times10^7$	& bits
	\end{tabular}
\end{table}

For this part we are once again going to inject bit flip into various sampling
of our density function. Each time we are going to inject various number of bit
flips (up to $10^6$ independant bit flips) and then evaluate the difference to
the expected range.

%=====================================================================
%=====================================================================
\ifstandalone
	\bibliographystyle{apalike}
	\bibliography{\rootPath Annexes/biblio}
\fi
%=====================================================================
%=====================================================================
\end{document}
