\documentclass[10pt,a4paper,twoside,twocolumn]{article}
%% Lots of packages !
\usepackage{etex}

%% Francisation
\usepackage[english]{babel}
\usepackage[T1]{fontenc}
\usepackage[utf8]{inputenc}
%\usepackage{textcomp}

%% Réglages généraux
\usepackage[left=1.5cm,right=1.5cm,top=2cm,bottom=2cm]{geometry}
\usepackage{fancyhdr}
\usepackage{setspace}
\usepackage{lscape}
%\usepackage{multicol}
\usepackage{makeidx}
\usepackage[clearempty]{titlesec}
\usepackage{cite}

%% Packages pour le texte
\usepackage{pifont}
\usepackage{eurosym}
\usepackage{soul}
\usepackage[normalem]{ulem}
\usepackage{fancybox}
\usepackage{boxedminipage}
\usepackage{enumerate}
\usepackage{verbatim}
\usepackage{moreverb}
\usepackage{listings}
\usepackage[table]{xcolor}

%% Packages pour les tableaux
\usepackage{array}
\usepackage{multirow}
\usepackage{tabularx}
\usepackage{longtable}

%% Packages pour les dessins
\usepackage{graphicx}
\usepackage{wrapfig}
%\usepackage{picins}
\usepackage{picinpar}
\usepackage{epic}
\usepackage{eepic}
\usepackage{tikz}
\usepackage{afterpage}
\usepackage{rotating}
\usepackage{float}
\usepackage{caption}

%% Packages pour les maths
\usepackage{amsmath}
\usepackage{amssymb}
\usepackage{dsfont}
\usepackage{mathrsfs}
\usepackage{bussproofs}
\usepackage[thmmarks,amsmath]{ntheorem}

%% Création de nouvelles commandes
%\usepackage{calc}
\usepackage{ifthen}
\usepackage{xspace}



\usepackage{url}
\usepackage{hyperref}
\usepackage{todonotes}
\usepackage{subcaption}
\usepackage[french,ruled,vlined,linesnumbered,algosection,dotocloa]{algorithm2e}
\usepackage{MnSymbol}

\usepackage{chngcntr}

\usepackage{standalone}
\usepackage{import}

\usepackage[affil-it]{authblk}


\usepackage{lipsum}












\numberwithin{equation}{subsection}


\newcommand*{\rootPath}{../}
\standalonetrue

\begin{document}


\section{Density analysis tools}

In order to compare different density fields produced by the density estimation
methods, we need metrics caracterising the representative elements of those
density fields. We hereby discuss different methods for evaluating density
fields discrepency.

\subsection{Power spectrum based integral}

The first thought one could have would be to try building methods which
differentiate values according to the relevant elements in regard to their
futher analysis in real application.

For this step, the use a power spectrum analysis which makes visible variation
responses for different caracteristic length, hence detecting bias such as high
frequency noise or over smoothness. In order to compare distances between those
spectral responces we use the following integral based metric :

\begin{equation}
	d(ps_1, ps_2) = \int_{\Delta f}\frac{\left\|\log\left(\frac{ps_1(f)}{ps_2(f)}
										\right)\right\|}{f^2}\, \mathrm df
	\label{eq:psd_metric}
\end{equation}

This integral considers the difference of magnitude of the considered responces.
That way we have a value which is consistent regardless of the general shape of
the reponces. Yet this gives a very high weight to high frequencies, which is
reduced by dividing this ratio value by $f^2$.

This division can be seen as $2$ divisions by $f$, the first one done in orther
to have constant integral weight over each order or magnitude one the $x$ axis.
Therefore the same contribution comes from all order of magnitudes, removing the
higher contribution of higher frequencies. The second division acts as a low
pass filter, which helps differentiate spectral responce who diverge from one
another in the lower range.

\subsection{Distribution based stastical metrics}

We also considered other metrics, looking the density field as a distribution
function. We therefore used to widely used statistical matrics.

\subsubsection{Helligner distance}
\begin{equation}
	H(P, Q) = \frac{1}{\sqrt{2}}\sqrt{\sum_i \sqrt{p_i} - \sqrt{q_i}}
\end{equation}

\subsubsection{Bhattacharyya distance}
\begin{equation}
	D_B(P, Q) = -\ln{\left(\sum_i \sqrt{p_i \times q_i}\right)}
\end{equation}




% \begin{figure}[!ht]
% 	\centering
% 	\includegraphics[width=0.48\textwidth]
% 		{\rootPath Figures/synthetic/Stat-divergence.pdf}
% 	% \caption{Spectral analysis of density estimation methods processing synthetic
% 	% 	samplings from or analytical model}
% 	% \label{fig:synthetic:spectral}
% \end{figure}


\subsection{Geometry based metrics}


% \begin{figure}[!ht]
% 	\centering
% 	\includegraphics[width=0.48\textwidth]
% 		{\rootPath Figures/synthetic/kolmogorov-integral.pdf}
% 	% \caption{Spectral analysis of density estimation methods processing synthetic
% 	% 	samplings from or analytical model}
% 	% \label{fig:synthetic:spectral}
% \end{figure}


\begin{equation}
	K(P, Q) = \iint\limits_{x,y}\left\|
							\iint\limits_{(0,0)}^{(x,y)} P(i,j)-Q(i,j)\,\mathrm{d}i\,\mathrm{d}j
						\right\|\,\mathrm{d}x\,\mathrm{d}y
\end{equation}







%=====================================================================
%=====================================================================
\ifstandalone
	\bibliographystyle{apalike}
	\bibliography{\rootPath Annexes/biblio}
\fi
%=====================================================================
%=====================================================================
\end{document}
